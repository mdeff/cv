\documentclass[11pt,a4paper,sans]{moderncv}
\moderncvstyle{casual}  % casual, classic, oldstyle, banking
\moderncvcolor{blue}  % blue, orange, green, red, purple, grey, black

% TODO: style (less color, more minimal) and fonts (Adobe Source?).

%pdflatex
%\usepackage[T1]{fontenc}
%\usepackage[utf8]{inputenc}  % default since texlive 2018

%xelatex and lualatex
%\usepackage{fontspec}
%\defaultfontfeatures{Ligatures=TeX}
%\setmainfont{Linux Libertine O}
%\setsansfont{Linux Biolinum O}

% using academicons (for Google Scholar link) requires fontspec which requires XeLaTeX or LuaLaTeX
\usepackage{iftex}
\usepackage{academicons}

\usepackage[scale=0.75]{geometry}  % reduce margins
\setlength{\hintscolumnwidth}{2.25cm}  % width of the dates column

\usepackage[authorlastname=Defferrard, startyear=2000]{autopub}
\addbibresource{publications.bib}

\usepackage{csquotes}  % quote with \enquote{}

%\newcommand{\httpslink}[1]{\href{https://#1}{#1}}  % better footnote, worse inline
\renewcommand{\httpslink}[1]{\href{https://#1}{\texttt{#1}}}
\newcommand*{\spacing}{0.5em}  % add some whitespace

%-------------------------------------------------------------------------------

% definition and usage in /usr/share/texmf-dist/tex/latex/moderncv/moderncv.cls

%\name{Michaël Defferrard}{(mdeff)}
\name{Michaël}{Defferrard}
%\title{Curriculum Vitae}
%\title{mdeff}
%\photo[70pt][0.4pt]{picture}

%\quote{\enquote{I make machines learn on unstructured data.}}
%\quote{\enquote{I make machines learn on data structured by graphs.}}
%\quote{\enquote{I make machines learn; better with graphs.}}
\quote{\enquote{I make machines learn; better by leveraging structure.}}
%\quote{\enquote{I make machines learn; with structured data.}}
%\quote{\enquote{I make machines learn, and see graphs everywhere.}}
%\quote{\enquote{I make machines learn from structured data.}}
%\quote{\enquote{Research on Machine Learning and Graphs.}}

% TODO: how much personal details here? I'd say the least.
% no nationality, birthday, photo, phone
% TODO: most important links (website, scholar, github, twitter) at top?
%\address{EPFL}{Switzerland}
%\phone{+41 00 000 00 00}  % or \phone[mobile]
\email{michael.defferrard@epfl.ch}
%\email{michael@deff.ch}
\homepage{deff.ch}
%\social[linkedin]{mdeff}
%\social[twitter]{m\_deff}
%\social[github]{mdeff}
%\social[googlescholar]{Ztj2-gUAAAAJ}
%\social[orcid]{0000-0002-6028-9024}
%\social[arxiv]{0000-0002-6028-9024}
%\social[youtube]{}
\collectionadd[noicon]{socials}{
    \protect\href{https://linkedin.com/in/mdeff}{\small\faLinkedin}~
    \protect\href{https://twitter.com/m_deff}{\small\faTwitter}~
    \protect\href{https://github.com/mdeff}{\small\faGithub}~
    %\protect\href{https://orcid.org/0000-0002-6028-9024}{\small\aiOrcid}~
    %\protect\href{https://arxiv.org/a/0000-0002-6028-9024.html}{\small\aiarXiv}~
    \ifxetexorluatex  % icon font requires fontspec
        \protect\href{https://scholar.google.com/citations?user=Ztj2-gUAAAAJ}{\small\aiGoogleScholar}
    \fi
}
\def\today{\ifcase\month\or  % month and year
  January\or February\or March\or April\or May\or June\or
  July\or August\or September\or October\or November\or December\fi
  \space \number\year}
\extrainfo{updated \today, latest at \httpslink{deff.ch/cv.pdf}}

%-------------------------------------------------------------------------------

\begin{document}

\makecvtitle

%\section{Objective}  % Objective(s), About, Summary, Profile
% Research interests, Life goals
% My research interests revolve around ML, graphs, geometry. I enjoy most to create solutions that solve real problems.
% TODO: here or as the \quote{}? Better in a research statement / cover letter?
% Not common on academic CVs. But I think import when applying.

\section{Strengths} % Summary, Objective
% Only for a targeted (to a job opening) CV. Not for the generic CV on my website. Or for a generic CV to be circulated as the sole document.

% My story. On website (generic) and cover letter (targeted).
%\vspace{-0.5cm}
%I am a Machine Learning researcher interested in the modeling, analysis, and understanding of structured data.
%Data structured by networks, such as brain activity supported by neural connections, or manifolds, such as the temperature and wind fields on the Earth.
%To this end I am developing Deep Learning to exploit that structure, often modeled as a graph.
%In my research, I draw from theoretical insights and practical needs to develop principled methods---and strive for impactful applications by co-leading interdisciplinary research efforts.

\begin{itemize}  % hard skills (not soft, which are demonstrated in letter and interview)
%    \item Knowledgeable and passionate about structure and learning. % kind of like my headline
%    \item Experienced with graph neural networks and related methods.
    \item Experienced with Machine/Deep Learning on graphs/networks/manifolds/meshes/complexes.
%    \item Experienced with Neural Networks on graphs/networks/manifolds/meshes/complexes.
%    \item Experienced with ML for protein imaging and design.
    \item Skilled in software development (ML frameworks, scientific Python, package maintenance).
%        % both using ML frameworks and maintaining Python packages
%        % scientific Python, (ML frameworks, scientific Python, package maintenance)
%    \item Computing enthusiast with interest and knowledge from chips to OSs and toolchains to apps.
            % mastery of the stack is paramount to efficient AI/ML (full-stack)
    \item Published at top ML and domain-specific venues and contributed to open-source software.
%    % \item Geometry, calculus
%    % \item I draw from theoretical insights and practical needs to develop principled methods---and strive for impactful applications by co-leading interdisciplinary research efforts.
    % \item State impact (pioneering work, impact on production, citations, OSS usage). But how to quantify and state it concisely?
\end{itemize}

% (Key) Skills (computer skills, technical skills, communication skills (writings?)

\section{Education}
% content: credentials (and experience)

\cventry[\spacing]{2015–2021}{PhD Machine Learning}{\href{https://www.epfl.ch}{École Polytechnique Fédérale de Lausanne (EPFL)}}{}{}{%
\begin{itemize}
    \item Thesis: \href{https://infoscience.epfl.ch/record/294301}{Leveraging topology, geometry, and symmetries for efficient Machine Learning.}
    % \item Topics: machine learning and signal processing on graphs, symmetries and equivariance, topology and geometry, graph neural networks.
        % deep learning and graphs, geometric deep learning
    \item Adviser: Prof.\ \href{https://people.epfl.ch/pierre.vandergheynst}{Pierre Vandergheynst}.
    \item Examiners: \href{https://people.epfl.ch/martin.jaggi}{Martin Jaggi} (EPFL), \href{https://staff.fnwi.uva.nl/m.welling}{Max Welling} (UvA, MSR), \href{http://yann.lecun.com}{Yann LeCun} (NYU, FAIR).
\end{itemize}
}
\cventry[\spacing]{2012–2015}{MSc Electrical and Electronic Engineering}{\href{https://www.epfl.ch}{EPFL}}{}{\textit{GPA 96\%}}{%
\begin{itemize}
    % \item Focus in Information Technologies.
    \item Thesis: \href{https://infoscience.epfl.ch/record/218019}{Structured Auto-Encoder with application to Music Genre Recognition.} %Advisors: Xavier Bresson, Johan Paratte, Pierre Vandergheynst.
    \item Minor in Computational Neuroscience.
    \item Courses and projects on signal processing, data analysis, machine learning.
\end{itemize}
}
\cventry[\spacing]{2009–2012}{BASc Electrical Engineering}{\href{https://www.heia-fr.ch}{École d'Ingénieurs de Fribourg (EIA-FR)}}{}{\textit{GPA 98\%}}{%
% École d'ingénieurs et d'architectes de Fribourg (EIA-FR)
% Haute école d'ingénierie et d'architecture Fribourg (HEIA-FR)
% School/College of Engineering and Architecture of Fribourg
\begin{itemize}
    \item Courses and projects on electronic design, analog and digital circuits, embedded systems.
%    * signal processing, programming?, telecommunication, system on chip
%    \item Low-level programming (assembly, microcontroller, Linux kernel drivers) for embedded systems.
%    I have basic and rusty experience in HW description with VHDL, HW simulators, logic circuits, transistor design and placement, etc.
%    I used to program embedded systems and SoCs. Bare metal (assembly and C) and Linux kernel drivers.
%    \item Developed a robot positioning system for the Eurobot competition.
%    \item Major in Electronics. % Orientation Électronique.
%    \item French and German bilingual studies.
    \item Exchange year at the \href{https://hm.edu}{Hochschule München}. % (Erasmus)
        % Fachhochschule / Hochschule / University of Applied Sciences
%    \item Phonak Communications award for excellence.
\end{itemize}
}
%\cventry[\spacing]{2010--2011}{Exchange student (ERASMUS)}{Fachhochschule München}{}{}{}
\cventry[\spacing]{2005–2009}{Federal VET Diploma in Electronics}{\href{https://www.fr.ch/epai}{EPAI}}{Fribourg CH}{\textit{GPA 96\%}}{%
% Vocational Education and Training (VET)
% Certificat Fédéral de Capacité (CFC)
% Ecole professionnelle artisanale et industrielle de Fribourg
%\begin{itemize}
%    \item Professional Technical Maturity Certificate
%    \item Award for excellence from Union Patronale du Canton de Fribourg (UPCF)
%\end{itemize}
}

\section{Experience}
% content: what I did, and the impact it had

\cventry[\spacing]{present\\2022-08}{Research Scientist (Staff Engineer)}{\href{https://www.qualcomm.com/research}{Qualcomm Research}}{remote}{}{%
On the fundamental and curiosity-driven side, I want to leverage the fundamental principles that underpin our reality—like topology, geometry, symmetries, causality—for machines to learn better. % more efficiently
On the applied and problem-driven side, I am researching how that can help the design of semiconductor chips;
% IC/SoC, semiconductor chips, EDA, electronic/chip/IC design, design of system-on-chips (SoCs)
% Physical chip design (from abstract circuit design to geometric chip layout) as a graph transformation (netlist optimizations) and embedding (place and route) problem. Similarly for compilers. Overall: abstract computation graph to concrete (physical) implementation.
in collaboration with product partners and academics. % internal and external
}
\cventry[\spacing]{2021-12\\2014-02}{Research Assistant}{\href{https://www.epfl.ch}{École Polytechnique Fédérale de Lausanne (EPFL)}}{}{}{%
%Doctoral assistant (and previously project student) at the LTS2 laboratory led by Prof.\ Pierre Vandergheynst.
% I researched on Machine Learning and data structured by graphs and manifolds.
I worked toward a fundamental understanding of graphs, researched how to leverage that structure for machines to learn better, % more efficiently, regarding computational, memory, and data resources.
and applied it in Neuroscience, Cosmology, Biology, Geoscience. % AI4Science, —for cosmological inference, protein imaging and design, brain understanding, weather forecasting.
I published papers in top-tier venues, co-led interdisciplinary research teams, supervised students, gave talks, taught courses, developed software.
My work pioneered graph ML research, proved useful in tackling important problems, and was nominated—by \href{https://people.epfl.ch/pierre.vandergheynst}{Pierre Vandergheynst}, \href{https://people.epfl.ch/martin.jaggi}{Martin Jaggi}, \href{https://staff.fnwi.uva.nl/m.welling}{Max Welling}, \href{http://yann.lecun.com}{Yann LeCun}—for the EPFL's best PhD thesis award.
% more specific: GNNs, GDL, relational DL; weather forecasting, cosmology
%impact of work: citations, enabled countless applications, etc.
% specific impact: the head of ML research at the European Centre for Medium-Range Weather Forecasts (ECMWF) wants to integrate the developed method in their operational forecasting system. That shall improve weather services and extreme events preparedness for businesses across multiple industries and billions of individuals.
}
\cventry[\spacing]{2015-08\\2011-08}{Software Engineer}{\href{https://infoteam.ch}{Infoteam}}{Givisiez CH}{}{%
% \href{https://elvexys.com}{Elvexys} broke of \href{https://infoteam.ch}{Infoteam} in 2019 and was sold to condis in 2021
Part-time job in the Energy R\&D team.
I ported a core product of the company, a control-command tool for energy distribution and production, %called StreamX,
to embedded systems.
My work enabled the company to close its largest contract to date.
}
\cventry[\spacing]{2012-08\\2012-05}{Research Intern}{\href{https://www.lbl.gov}{Lawrence Berkeley National Laboratory (LBNL)}}{}{}{%
%Bachelor thesis. %award for excellence.
I characterized the performance of a new particle detector for the ATLAS experiment at the CERN's Large Hadron Collider (LHC).
}
\cventry[\spacing]{2011-03\\2005-08}{Electronics Specialist}{\href{https://www.meggitt.com}{Meggitt}}{Fribourg CH}{}{%
Apprenticeship and part-time job.
Production, test, quality assurance, repair, certification and development of sensing systems for the aerospace and energy markets.
%Required the respect of strict/high quality standards.
}

\section{Awards}  % Awards, Scholarships, Achievements, Accomplishments, Honours, Fellowships, Recognition
\cvitem{2021}{Nominated for the \href{https://www.epfl.ch/education/phd/phd-awards/}{EPFL Doctorate award}. Given to the best $2$ of $\sim 420$ theses.}
% The EPFL Doctorate Award was established in 1993 to distinguish the work performed during an outstanding doctoral thesis at the EPFL, and to encourage particularly qualified researchers. The award honors a candidate who performed a remarkable thesis as to its originality, the impact of the results (publication(s) in one or several international journals) and presentation of the work.
% https://www.epfl.ch/research/awards/epfl-research-awards/
% https://www.epfl.ch/education/phd/phd-awards/
% PhD theses per year (2021-2016): (435+400+422+408+430+406)/6 = 417 <https://infoscience.epfl.ch/search?cc=Infoscience%2FResearch%2FThesis>
% TODO: exact number of theses between oct 2021-2022
\cvitem{2020}{Spotlight talk for~\cite{deepsphere_iclr} at \href{https://iclr.cc}{ICLR}.}
% TODO: (given to the) top x% of papers
\cvitem{2016, 2017}{Nominated by EPFL for a \href{https://research.google/outreach/phd-fellowship}{Google PhD Fellowship}.}
\cvitem{2014}{Selected and funded for the \href{https://www.bcv.ch/La-BCV/Responsabilite-d-entreprise/RSE/Silicon-Valley-Startup-Camp}{Silicon Valley Startup Camp}.}
% a week-long trip to get exposed to and learn about entrepreneurship
\cvitem{2012}{Award from \href{https://www.phonak-communications.com}{Phonak Communications} for an excellent BASc thesis.}
% Sept 2012, Par ce prix, Phonak Communications exprime la reconnaissance de l'excellence du travail réalisé lors du projet de Bachelor effectué à l'école d'ingénieurs et d'architectes de Fribourg en été 2012.
\cvitem{2009}{Awards from the \href{https://www.upcf.ch}{UPCF} and the \href{https://www.fr.ch/dee/sfp}{SFP} for the highest GPA.}
%\cvitem{2009}{Highest GPA among all technical professions, \href{https://www.upcf.ch}{Union Patronale du Canton de Fribourg}.}
% <https://unionpatronale.ch> -> <https://www.upcf.ch>
% \cvitem{2009}{Award from the \href{https://www.fr.ch/dee/sfp}{Service de la formation professionnelle} for the highest GPA in my profession.}
% \href{https://www.fr.ch/dee/sfp}{Service de la formation professionnelle (SFP), Direction de l'économie et de l'emploi (DEE), Canton/Etat de Fribourg}

%\section{Grants}  % Grants, Funding
% No successful grants yet.

%\section{Talks}
% List here or on website. Boring to repeat.
% Probably more visible on website, with links to resources.
% Mention in Miscellaneous.

%\section{Teaching}
% Same as talks.

% \section{Languages}
% \cvlanguage{english}{advanced}{CAE grade A (2012)}
% Not really useful for academic CV, as English is the expected working language.
% \cvitem{French}{Mothertongue}
% \cvitem{English}{Fluent (C2)}
% \cvitem{German}{Intermediate (B2)}

%%%%%%%%%%%%%%%%%%%%%%%%%%%%%%%%%%%%%%%%%%%%%%%%%%%%%%%%%%%%%%%%%%%%%%%%
% things not worth their own section
% or Scientific Output, with full list of publications online only
% TODO: what should be promoted to its own section, and what should stay here?
%    * own section if listed with full details
%    * here if summarized
% then, what should be listed in full details, what should be summarized?
% Ideally most things are on website (as a portfolio). To avoid overlap, CV has history (experience), credentials (education, awards), and summary (statistics) (pubs, teaching, supervision, etc.).
%%%%%%%%%%%%%%%%%%%%%%%%%%%%%%%%%%%%%%%%%%%%%%%%%%%%%%%%%%%%%%%%%%%%%%%%

\section{Scientific output}

%\cventry[\spacing]{Publications}{10+ papers, 4500+ citations, h-index of 10}{}{}{}{%
\cvitem{Publications}{%
    \textbf{10+ papers, 7000+ citations, h-index of 10.}
    Published at top ML conferences (NeurIPS, ICLR) as well as domain-specific journals (NeuroImage, Astronomy \& Computing) and conferences (The Web Conference, ISMIR).
    List below, on \href{https://scholar.google.com/citations?user=Ztj2-gUAAAAJ}{Google Scholar}, and at \httpslink{deff.ch} with code, data, reviews. % slides, blog, poster, video
}
%\cvitem{Patents}{}
%\cventry[\spacing]{Software}{Working on the open-source scientific Python stack}{}{}{}{%
\cvitem{Software}{%
    Maintainer of 3 Python packages. %, downloaded x+ times.
    Contributor to the Python scientific stack (NumPy, SciPy, Matplotlib, Jupyter, etc.).
    List below, with more open-source contributions (paper implementations, teaching materials) at \httpslink{github.com/mdeff}.
    % contributions to the python scientific stack and jupyter
}
\cvitem{Talks}{I gave 20+ talks. List with slides and some videos at \httpslink{deff.ch}.}
% TODO: ORCID could list publications, talks, etc.
\cvitem{Teaching}{I co-taught 8 courses around Machine Learning, Networks, and Data Science, in various roles (TAing, lecturing, teaching team \& student management, curriculum design) and forms (university class, workshop, summer school). List with roles and resources at \httpslink{deff.ch}.}
% Supervision (managed), Advizing (gave advices), Mentoring (informal?)

\section{Leadership}
% leadership: supervision, organization, teaching

\cvitem{Supervision}{I supervised 20+ students (MSc theses, semester projects, internships). List of students with project title, co-supervisors, and outputs at \httpslink{deff.ch}.}
\cvitem{Collaborations}{I tackled difficult problems by assembling and leading interdisciplinary collaborations.}
% real-world problems sound week → difficult problems / (grand) challenges
% So far in Neuroscience, Cosmology, Biology, Geoscience, chip design/EDA. → in Experience
% by looking for problems and assembling interdisciplinary collaborations to tackle them.
% with applications downstream, with math upstream
\cvitem{Organization}{%
    \listitemsymbol{} \href{https://www.ml4earth.org}{Machine Learning for Earth}, Seminar, 2019--2020. \newline
    \listitemsymbol{} \href{https://www.crowdai.org/challenges/www-2018-challenge-learning-to-recognize-musical-genre}{Musical Genre Recognition Challenge}, The Web Conference (WWW), 2018. \newline
    \listitemsymbol{} \href{https://osip2017.epfl.ch}{Open Science in Practice}, Summer School, EPFL, 2017. \newline
    \listitemsymbol{} \href{https://bmvc2017.london/dlid}{Deep Learning on Irregular Domains}, Workshop, BMVC, 2017.
}
\cvitem{Extra}{I serve as the president of a \href{https://fanfare-orsonnens.ch}{band of 40 musicians} and a leadership member at a \href{https://villorsonnens.ch/service-technique/service-du-feu}{firefighting brigade of 80}.}
% commanding officer: (experienced in) taking rapid decisions under pressure (when a fire breaks out)
% Vice-president of a music festival (committee of 14, xx in sub-committees, 1,000 volunteers, 10,000 visitors, financial xx revenue, benefits).

\section{Miscellaneous}

% \cvitem{Reviewing}{
%     IEEE Transactions on Medical Imaging (TMI),  % 2019
%     IEEE Transactions on Image Processing (TIP),  % 2019
%     IEEE Transactions on Pattern Analysis and Machine Intelligence (TPAMI),  % 2017
%     IEEE Transactions on Neural Networks and Learning Systems (TNNLS),  % 2017
%     IEEE Global Conference on Signal and Information Processing (GlobalSIP),  % 2017
%     IEEE Journal of Selected Topics in Signal Processing (J-STSP).  % 2016
% }
% Professional Services: reviewer, editor, chair, organizer?, thesis committees?
\cvitem{Open}{Open science, open source, open data, and reproducibility are values I advocate for and adhere to in my research.}
% Everything should be public. Value clear explanations / writing.
%\cvitem{Communication}{I value and practice clear writing.}
\cvitem{Extra}{Brass band musician, militia firefighting officer, runner, computing enthusiast.}

%\section{Extra-Curricular}  % Interests, Extra-Curricular Activities
% TODO: Keep or remove? In Miscellaneous?
%\cvlistitem{Brass Band musician, playing in two bands at the Swiss national championship}
%\cvlistitem{Firefighter officer in the local militia}
%\cvlistitem{Computing systems and open-source enthusiast}
%\cvlistitem{Interests in entrepreneurship and start-ups}

%\section{Online}
% Only useful for short form without full listings (those links are otherwise prominently mentioned already).
% TODO: but shall I maintain a short form?
%\cvitemwithcomment{}{\httpslink{deff.ch}}{Publications, talks, teaching, advizing, software, projects.}
%\cvitemwithcomment{}{\httpslink{github.com/mdeff}}{Open source contributions.}
%\cvitemwithcomment{}{\httpslink{scholar.google.com/citations?user=Ztj2-gUAAAAJ}}{Publications.}
%\cvitemwithcomment{}{\httpslink{twitter.com/m_deff}}{Thoughts.}
% orcid, arxiv, zenodo, youtube, linkedin

%%%%%%%%%%%%%%%%%%%%%%%%%%%%%%%%%%%%%%%%%%%%%%%%%%%%%%%%%%%%%%%%%%%%%%%%
% Appendix
\clearpage
%%%%%%%%%%%%%%%%%%%%%%%%%%%%%%%%%%%%%%%%%%%%%%%%%%%%%%%%%%%%%%%%%%%%%%%%

\section{Publications \hfill \small \httpslink{scholar.google.com/citations?user=Ztj2-gUAAAAJ}}
\autopuball
%\autopubbytype
% Google Scholar h-index of 5, 2000+ citations.
% TODO: add approximate citation counts?

%\section{Selected Publications \hfill \small \httpslink{scholar.google.com/citations?user=Ztj2-gUAAAAJ}}
%\addtocategory{selected}{snn}
%\addtocategory{selected}{deepsphere_iclr}
%\addtocategory{selected}{deepsphere_cosmo}
%\addtocategory{selected}{fma_dataset}
%\addtocategory{selected}{cnn_graph}
%\autopubselected

\section{Software \hfill \small \httpslink{github.com/mdeff}}
\autopubsoftware
%\section{Selected Software \hfill \small \httpslink{github.com/mdeff}}
%\addtocategory{selectedsoftware}{pygsp}
%\autopubselectedsoftware
% TODO: usage metrics? (downloads, stars)

\end{document}

% ---------------------------------------------------------

% \section{Features}
% from /usr/share/texmf-dist/tex/latex/moderncv/moderncv.cls

% \section{title}
% \subsection{title}

% \cventry[\spacing]{2000--2010}{degree/title}{institution}{where}{grade}{%
% comment
% }

% \cvitem[1em]{header}{text}
% \cvitemwithcomment[1em]{header}{text}{comment}
% \cvdoubleitem[1em]{header1}{text1}{header2}{text2}
% \cvlistitem{item}
% \cvlistdoubleitem{item1}{item2}

% \begin{cvcolumns}
%     \cvcolumn[0.4]{head}{content}
%     \cvcolumn[0.4]{head}{content}
% \end{cvcolumns}
