\documentclass[11pt,a4paper,sans]{moderncv}
% todo: better font

\moderncvstyle{casual}  % casual, classic, oldstyle, banking
\moderncvcolor{blue}  % blue, orange, green, red, purple, grey, black

%\usepackage[T1]{fontenc}
%\usepackage[utf8]{inputenc}

\usepackage[scale=0.75]{geometry}  % reduce margins
\setlength{\hintscolumnwidth}{2.3cm}  % width of the dates column

\usepackage[authorlastname=Defferrard, startyear=2000]{autopub}
\addbibresource{publications.bib}

%\newcommand{\httpslink}[1]{\href{https://#1}{#1}}  % better footnote, worse inline
\newcommand{\httpslink}[1]{\href{https://#1}{\texttt{#1}}}

%-------------------------------------------------------------------------------

% definition and usage in /usr/share/texmf-dist/tex/latex/moderncv/moderncv.cls

%\name{Michaël Defferrard}{(mdeff)}
\name{Michaël}{Defferrard}
%\title{Curriculum Vitae}
%\title{mdeff}

%\quote{``I make machines learn on unstructured data.''}
%\quote{``I make machines learn on data structured by graphs.''}
\quote{``I make machines learn; better with graphs.''}
%\quote{``I make machines learn, and see graphs everywhere.''}
%\quote{``Research on Machine Learning and Graphs.''}
% All things graphs
% My research interests revolve around ML, graphs, geometry. I enjoy most to create solutions that solve real problems.

% TODO: how much personal details here? I'd say the least.
% no nationality, birthday, photo, phone
% TODO: most important links (website, scholar, github, twitter) at top?
%\address{EPFL}{Switzerland}
%\phone{+41 00 000 00 00}  % or \phone[mobile]
\email{michael.defferrard@epfl.ch}
%\email{michael@deff.ch}
\homepage{deff.ch}
\social[linkedin]{mdeff}
\social[github]{mdeff}
\social[twitter]{m\_deff}
%\social[scholar]{Ztj2-gUAAAAJ}
%\social[youtube]{}
%\extrainfo{\href{https://scholar.google.com/citations?user=Ztj2-gUAAAAJ}{Scholar}}
\def\today{\ifcase\month\or  % month and year
  January\or February\or March\or April\or May\or June\or
  July\or August\or September\or October\or November\or December\fi
  \space \number\year}
\extrainfo{updated \today, latest at \httpslink{deff.ch/cv.pdf}}
%\photo[70pt][0.4pt]{picture}

\newcommand*{\spacing}{0.5em}  % add some whitespace

%-------------------------------------------------------------------------------

\begin{document}

\makecvtitle

%\section{Objective}  % Objective(s), About, Summary, Profile
% My research interests / Life goals
% todo: as the blue quote?

% (Key) Skills (computer skills, technical skills, communication skills (writings?)

\section{Education}
% content: credentials

\cventry[\spacing]{2015 -- present}{PhD candidate}{EPFL}{Lausanne CH}{}{%
%Research on machine learning and data structured by graphs. Advisor: Pierre Vandergheynst.}
Advized by Pierre Vandergheynst.
% \begin{itemize}
% 	\item Research: deep learning and graphs
% 		% geometric deep learning
% 	\item Advisor: Pierre Vandergheynst
% \end{itemize}
}
\cventry[\spacing]{2012 -- 2015}{MSc Electrical Engineering}{EPFL}{Lausanne CH}{\textit{GPA 96\%}}{%
Minor in Computational Neuroscience.
% \begin{itemize}
% 	%\item Focus in Information Technologies
% 	\item Minor in Computational Neuroscience
% 	%\item Thesis: Structured Auto-Encoder with application to Music Genre Recognition. Advisors: Xavier Bresson, Johan Paratte, Pierre Vandergheynst.
% \end{itemize}
}
\cventry[\spacing]{2009 -- 2012}{BASc Electrical Engineering}{EIA-FR}{Fribourg CH}{\textit{GPA 98\%}}{%
%\begin{itemize}
%	\item French and German bilingual studies
%	\item Phonak Communications award for excellence
%\end{itemize}
}
\cventry[\spacing]{2010 -- 2011}{ERASMUS}{Fachhochschule München}{Germany}{}{}
%\cventry[\spacing]{2005 -- 2009}{Federal Certificate of Capacity}{EPAI}{Fribourg CH}{\textit{GPA 96\%}}{%
%\begin[\spacing]{itemize}
%	\item Professional Technical Maturity Certificate
%	\item Award for excellence from Union Patronale du Canton de Fribourg (UPCF)
%\end{itemize}
%}

\section{Experience}
% content: what I did

\cventry[\spacing]{Feb 2014 -- present}{Research Assistant}{EPFL}{}{}{%
Doctoral assistant (and previously project student) at the LTS2 laboratory led by Prof. Pierre Vandergheynst.
Research on machine learning and data structured by graphs.
Besides research, I manage and co-teach a graduate course on data science with networks.
}
\cventry[\spacing]{2011 -- 2015}{Software Engineer}{Infoteam}{}{}{%
Software engineer in the Energy R\&D team.
My main contribution was to port a core product of the company, a control-command tool for energy distribution and production called StreamX, to embedded systems.
My work enabled the company to close its largest contract to date.
}

\cventry[\spacing]{May -- Aug 2012}{Research Intern}{LBNL}{Berkeley}{}{%
Bachelor thesis, award for excellence.
I characterized the performance of a particle detector for the Large Hadron Collider (LHC) at CERN.
}

\cventry[\spacing]{2005 -- 2011}{Electronics Specialist}{Meggitt}{Fribourg CH}{}{%
Apprenticeship and part-time job.
Production, test, quality assurance, repair, certification and development of sensing systems for the aerospace and energy markets.
%Required the respect of strict/high quality standards.
}

\section{Publications \hfill \small \httpslink{scholar.google.ch/citations?user=Ztj2-gUAAAAJ}}
\autopuball
%\autopubbytype
% Google Scholar h-index of 5, 2000+ citations.
% TODO: add approximate citation counts?

%\section{Selected Publications}
%\addtocategory{selected}{fma_dataset}
%\autopubselected

\section{Software \hfill \small \httpslink{github.com/mdeff}}
\autopubsoftware
More open-source contributions (e.g., paper implementations, teaching materials, contributions to the python scientific stack and jupyter) at \httpslink{github.com/mdeff}.
% todo: usage metrics? (downloads, stars)

\section{Awards}  % Awards, Scholarships, Achievements, Honours, Fellowships, Recognition
\cvitem{2020}{ICLR spotlight talk for \cite{deepsphere_iclr}}
\cvitem{2016, 2017}{Google PhD Fellowship Nominee}
\cvitem{2014}{Silicon Valley Startup Camp (selected and funded)}
\cvitem{2012}{Award for excellence (BSc thesis), Phonak Communications}
% Par ce prix, Phonak Communications exprime la reconnaissance de l'excellence du travail réalisé lors du projet de Bachelor effectué à l'école d'ingénieurs et d'architectes de Fribourg en été 2012.
\cvitem{2009}{Award for excellence (best GPA), Union Patronale du Canton de Fribourg}
% Best apprentice over all technical professions (state Fribourg).

%\section{Grants}  % Grants, Funding
% No successful grants yet.

%\section{Talks}
% List here or on website. Boring to repeat.
% Probably more visible on website, with links to resources.
% Mention in Miscellaneous.

%\section{Teaching}
% Same as talks.

% \section{Languages}
% \cvitem{French}{Mothertongue}
% \cvitem{English}{Fluent (C2)}
% \cvitem{German}{Intermediate (B2)}

\section{Miscellaneous}  % things not worth their own section
% or Scientific Output, with full list of publications online only
% todo: what should be promoted to its own section, and what should stay here?
%	* own section if listed with full details
%	* here if summarized
% then, what should be listed in full details, what should be summarized?
% Ideally most things are on website. To avoid overlap, CV has history (experience), credentials (education, awards), and summary (statistics) (pubs, teaching, advizing, etc.).

%\cvitem{Publications}{7 papers, x patents, Google Scholar h-index of 5, 2000+ citations.}
%\cvitem{Software}{Maintainer of 2 scientific Python packages, downloaded x+ times. Contributor of many others.}
\cvitem{Talks}{I gave 20+ talks. List with slides (and some videos) at \httpslink{deff.ch}.}
% TODO: ORCID could list publications, talks, etc.
\cvitem{Teaching}{I co-taught 8 courses around Machine Learning, Networks, and Data Science, in various roles (TAing, lecturing, teaching team \& student management, curriculum design) and forms (university class, workshop, summer school). List with roles and resources at \httpslink{deff.ch}.}
% Supervision, Advizing, Mentoring
\cvitem{Supervision}{I supervised 15+ students (MSc theses, semester projects, internships). List of students with project title and outputs at \httpslink{deff.ch}.}
% leadership: supervision, organization, teaching
\cvitem{Collaborations}{I enjoy to collaborate with domain experts. So far in Neuroscience, Cosmology, Protein Design \& Imaging, Weather \& Climate Sciences.}
% with applications downstream, with math upstream
\cvitem{Organization}{
	\begin{itemize}
		\item \href{https://www.crowdai.org/challenges/www-2018-challenge-learning-to-recognize-musical-genre}{Musical Genre Recognition Challenge}, The Web Conference (WWW), 2018
		\item \href{https://osip2017.epfl.ch}{Open Science in Practice Summer School}, EPFL, 2017
		\item\href{https://bmvc2017.london/dlid}{Deep Learning on Irregular Domains Workshop}, BMVC, 2017 \& 2018
	\end{itemize}
}
\cvitem{Reviewing}{
	IEEE Transactions on Medical Imaging (TMI),  % 2019
	IEEE Transactions on Image Processing (TIP),  % 2019
	IEEE Transactions on Pattern Analysis and Machine Intelligence (TPAMI),  % 2017
	IEEE Transactions on Neural Networks and Learning Systems (TNNLS),  % 2017
	IEEE Global Conference on Signal and Information Processing (GlobalSIP),  % 2017
	IEEE Journal of Selected Topics in Signal Processing (J-STSP)  % 2016
}
% Professional Services: reviewer, editor, chair, organizer?, thesis committees?

\section{Extra-Curricular}  % interests, extra-curricular activities
% In Miscellaneous?
\cvlistitem{Brass Band musician, playing in two bands at the Swiss national championship}
\cvlistitem{Firefighter officer in the local militia}
\cvlistitem{Computing systems and open-source enthusiast}
%\cvlistitem{Interests in entrepreneurship and start-ups}

%\section{Online}  % Miscellaneous, Online
% only useful for short form without full listings (those links are otherwise prominently mentioned already)
% but shall I maintain a short form?
%\cvitemwithcomment{}{\httpslink{deff.ch}}{Publications, talks, teaching, advizing, software, projects.}
%\cvitemwithcomment{}{\httpslink{github.com/mdeff}}{Open source contributions.}
%\cvitemwithcomment{}{\httpslink{scholar.google.com/citations?user=Ztj2-gUAAAAJ}}{Publications.}
%\cvitemwithcomment{}{\httpslink{twitter.com/m_deff}}{Thoughts.}
% orcid, arxiv, zenodo, youtube, linkedin


% ---------------------------------------------------------


% \section{Features}
% from /usr/share/texmf-dist/tex/latex/moderncv/moderncv.cls

% \section{title}
% \subsection{title}

% \cventry[\spacing]{2000 -- 2010}{degree/title}{institution}{where}{grade}{%
% comment
% }

% \cvitem[1em]{header}{text}
% \cvitemwithcomment[1em]{header}{text}{comment}
% \cvdoubleitem[1em]{header1}{text1}{header2}{text2}
% \cvlistitem{item}
% \cvlistdoubleitem{item1}{item2}

% \begin{cvcolumns}
% 	\cvcolumn[0.4]{head}{content}
% 	\cvcolumn[0.4]{head}{content}
% \end{cvcolumns}

\end{document}
