\documentclass[11pt,a4paper,sans]{moderncv}
\moderncvstyle{casual}  % casual, classic, oldstyle, banking
\moderncvcolor{blue}  % blue, orange, green, red, purple, grey, black
% TODO: try other fonts

%pdflatex
%\usepackage[T1]{fontenc}
%\usepackage[utf8]{inputenc}  % default since texlive 2018

%xelatex and lualatex
%\usepackage{fontspec}
%\defaultfontfeatures{Ligatures=TeX}
%\setmainfont{Linux Libertine O}
%\setsansfont{Linux Biolinum O}

% using academicons (for Google Scholar link) requires fontspec which requires XeLaTeX or LuaLaTeX
\usepackage{iftex}
\usepackage{academicons}

\usepackage[scale=0.75]{geometry}  % reduce margins
\setlength{\hintscolumnwidth}{2.3cm}  % width of the dates column

\usepackage[authorlastname=Defferrard, startyear=2000]{autopub}
\addbibresource{publications.bib}

%\newcommand{\httpslink}[1]{\href{https://#1}{#1}}  % better footnote, worse inline
\newcommand{\httpslink}[1]{\href{https://#1}{\texttt{#1}}}
\newcommand*{\spacing}{0.5em}  % add some whitespace

%-------------------------------------------------------------------------------

% definition and usage in /usr/share/texmf-dist/tex/latex/moderncv/moderncv.cls

%\name{Michaël Defferrard}{(mdeff)}
\name{Michaël}{Defferrard}
%\title{Curriculum Vitae}
%\title{mdeff}
%\photo[70pt][0.4pt]{picture}

%\quote{``I make machines learn on unstructured data.''}
%\quote{``I make machines learn on data structured by graphs.''}
\quote{``I make machines learn; better with graphs.''}
%\quote{``I make machines learn, and see graphs everywhere.''}
%\quote{``Research on Machine Learning and Graphs.''}
% All things graphs
% My research interests revolve around ML, graphs, geometry. I enjoy most to create solutions that solve real problems.

% TODO: how much personal details here? I'd say the least.
% no nationality, birthday, photo, phone
% TODO: most important links (website, scholar, github, twitter) at top?
%\address{EPFL}{Switzerland}
%\phone{+41 00 000 00 00}  % or \phone[mobile]
\email{michael.defferrard@epfl.ch}
%\email{michael@deff.ch}
\homepage{deff.ch}
%\social[linkedin]{mdeff}
%\social[twitter]{m\_deff}
%\social[github]{mdeff}
%\social[googlescholar]{Ztj2-gUAAAAJ}
%\social[orcid]{0000-0002-6028-9024}
%\social[arxiv]{0000-0002-6028-9024}
%\social[youtube]{}
\collectionadd[noicon]{socials}{
	\protect\href{https://linkedin.com/in/mdeff}{\small\faLinkedin}~
	\protect\href{https://twitter.com/m_deff}{\small\faTwitter}~
	\protect\href{https://github.com/mdeff}{\small\faGithub}~
	%\protect\href{https://orcid.org/0000-0002-6028-9024}{\small\aiOrcid}~
	%\protect\href{https://arxiv.org/a/0000-0002-6028-9024.html}{\small\aiarXiv}~
	\ifxetexorluatex  % icon font requires fontspec
		\protect\href{https://scholar.google.ch/citations?user=Ztj2-gUAAAAJ}{\small\aiGoogleScholar}
	\fi
}
\def\today{\ifcase\month\or  % month and year
  January\or February\or March\or April\or May\or June\or
  July\or August\or September\or October\or November\or December\fi
  \space \number\year}
\extrainfo{updated \today, latest at \httpslink{deff.ch/cv.pdf}}

%-------------------------------------------------------------------------------

\begin{document}

\makecvtitle

%\section{Objective}  % Objective(s), About, Summary, Profile
% My research interests / Life goals
% TODO: here or as the \quote{}? Better in a research statement / cover letter?

% (Key) Skills (computer skills, technical skills, communication skills (writings?)

\section{Education}
% content: credentials

\cventry[\spacing]{2015 -- present}{PhD candidate}{École Polytechnique Fédérale de Lausanne (EPFL)}{}{}{%
%Research on machine learning and data structured by graphs. Advisor: Pierre Vandergheynst.}
Advized by Prof.\ Pierre Vandergheynst.
% \begin{itemize}
% 	\item Research: deep learning and graphs
% 		% geometric deep learning
% 	\item Thesis: XXX
% 	\item Advisor: Pierre Vandergheynst
% \end{itemize}
}
\cventry[\spacing]{2012 -- 2015}{MSc Electrical Engineering}{EPFL}{}{\textit{GPA 96\%}}{%
%Signal processing and information technologies.
%Courses on signal processing, data analysis, machine learning.
Minor in Computational Neuroscience.
% \begin{itemize}
% 	%\item Focus in Information Technologies
% 	\item Minor in Computational Neuroscience
% 	%\item Thesis: Structured Auto-Encoder with application to Music Genre Recognition. Advisors: Xavier Bresson, Johan Paratte, Pierre Vandergheynst.
% \end{itemize}
}
\cventry[\spacing]{2009 -- 2012}{BASc Electrical Engineering}{École d'Ingénieurs de Fribourg (EIA-FR)}{}{\textit{GPA 98\%}}{%
%\begin{itemize}
%	\item French and German bilingual studies
%	\item Phonak Communications award for excellence
%\end{itemize}
}
\cventry[\spacing]{2010 -- 2011}{Exchange student (ERASMUS)}{Fachhochschule München}{}{}{}
%\cventry[\spacing]{2005 -- 2009}{Federal Certificate of Capacity in Electronics}{EPAI}{Fribourg CH}{\textit{GPA 96\%}}{%
%% Ecole professionnelle artisanale et industrielle de Fribourg
%\begin{itemize}
%	\item Professional Technical Maturity Certificate
%	\item Award for excellence from Union Patronale du Canton de Fribourg (UPCF)
%\end{itemize}
%}

\section{Experience}
% content: what I did
% could also group Education and Experience, under Positions and Education

\cventry[\spacing]{2014 -- present}{Research Assistant}{École Polytechnique Fédérale de Lausanne (EPFL)}{}{}{%
Doctoral assistant (and previously project student) at the LTS2 laboratory led by Prof.\ Pierre Vandergheynst.
Research on machine learning and data structured by graphs.
I published papers in top-tier venues, supervised students, gave talks, taught courses, developed software.
%Besides research, I manage and co-teach a graduate course on data science with networks.
}
\cventry[\spacing]{2011 -- 2015}{Software Engineer}{Infoteam}{Givisiez CH}{}{%
Part-time job in the Energy R\&D team.
I ported a core product of the company, a control-command tool for energy distribution and production, %called StreamX,
to embedded systems.
My work enabled the company to close its largest contract to date.
}
\cventry[\spacing]{May -- Aug 2012}{Research Intern}{Lawrence Berkeley National Laboratory (LBNL)}{}{}{%
%Bachelor thesis. %award for excellence.
I characterized the performance of a new particle detector for the ATLAS experiment at the CERN's Large Hadron Collider (LHC).
}
\cventry[\spacing]{2005 -- 2011}{Electronics Specialist}{Meggitt}{Fribourg CH}{}{%
Apprenticeship and part-time job.
Production, test, quality assurance, repair, certification and development of sensing systems for the aerospace and energy markets.
%Required the respect of strict/high quality standards.
}

\section{Publications \hfill \small \httpslink{scholar.google.ch/citations?user=Ztj2-gUAAAAJ}}
\autopuball
%\autopubbytype
% Google Scholar h-index of 5, 2000+ citations.
% TODO: add approximate citation counts?

%\section{Selected Publications}
%\addtocategory{selected}{fma_dataset}
%\autopubselected

\section{Software \hfill \small \httpslink{github.com/mdeff}}
\autopubsoftware
More open-source contributions (e.g., paper implementations, teaching materials, contributions to the python scientific stack and jupyter) at \httpslink{github.com/mdeff}.
% TODO: usage metrics? (downloads, stars)

\section{Awards}  % Awards, Scholarships, Achievements, Accomplishments, Honours, Fellowships, Recognition
\cvitem{2020}{ICLR spotlight talk for \cite{deepsphere_iclr}}
\cvitem{2016, 2017}{Google PhD Fellowship Nominee}
\cvitem{2014}{Silicon Valley Startup Camp (selected and funded)}
\cvitem{2012}{Award for excellence (BSc thesis), Phonak Communications}
% Sept 2012, Par ce prix, Phonak Communications exprime la reconnaissance de l'excellence du travail réalisé lors du projet de Bachelor effectué à l'école d'ingénieurs et d'architectes de Fribourg en été 2012.
\cvitem{2009}{Award for excellence (highest GPA), Union Patronale du Canton de Fribourg}
% Best apprentice over all technical professions (state Fribourg).

%\section{Grants}  % Grants, Funding
% No successful grants yet.

%\section{Talks}
% List here or on website. Boring to repeat.
% Probably more visible on website, with links to resources.
% Mention in Miscellaneous.

%\section{Teaching}
% Same as talks.

% \section{Languages}
% Not really useful for academic CV, as English is the expected working language.
% \cvitem{French}{Mothertongue}
% \cvitem{English}{Fluent (C2)}
% \cvitem{German}{Intermediate (B2)}

\section{Miscellaneous}  % things not worth their own section
% or Scientific Output, with full list of publications online only
% TODO: what should be promoted to its own section, and what should stay here?
%	* own section if listed with full details
%	* here if summarized
% then, what should be listed in full details, what should be summarized?
% Ideally most things are on website. To avoid overlap, CV has history (experience), credentials (education, awards), and summary (statistics) (pubs, teaching, supervision, etc.).

%\cvitem{Publications}{7 papers, x patents, Google Scholar h-index of 5, 2000+ citations.}
%\cvitem{Software}{Maintainer of 2 scientific Python packages, downloaded x+ times. Contributor to many others.}
\cvitem{Talks}{I gave 20+ talks. List with slides (and some videos) at \httpslink{deff.ch}.}
% TODO: ORCID could list publications, talks, etc.
\cvitem{Teaching}{I co-taught 8 courses around Machine Learning, Networks, and Data Science, in various roles (TAing, lecturing, teaching team \& student management, curriculum design) and forms (university class, workshop, summer school). List with roles and resources at \httpslink{deff.ch}.}
% Supervision (managed), Advizing (gave advices), Mentoring (informal?)
\cvitem{Supervision}{I supervised 15+ students (MSc theses, semester projects, internships). List of students with project title and outputs at \httpslink{deff.ch}.}
% leadership: supervision, organization, teaching
\cvitem{Collaborations}{I enjoy to collaborate with domain experts to solve real problems. So far in Neuroscience, Cosmology, Protein Design \& Imaging, Weather \& Climate Sciences.}
% with applications downstream, with math upstream
\cvitem{Organization}{
	\begin{itemize}
		\item \href{https://www.crowdai.org/challenges/www-2018-challenge-learning-to-recognize-musical-genre}{Musical Genre Recognition Challenge}, The Web Conference (WWW), 2018
		\item \href{https://osip2017.epfl.ch}{Open Science in Practice Summer School}, EPFL, 2017
		\item\href{https://bmvc2017.london/dlid}{Deep Learning on Irregular Domains Workshop}, BMVC, 2017 \& 2018
	\end{itemize}
}
\cvitem{Reviewing}{
	IEEE Transactions on Medical Imaging (TMI),  % 2019
	IEEE Transactions on Image Processing (TIP),  % 2019
	IEEE Transactions on Pattern Analysis and Machine Intelligence (TPAMI),  % 2017
	IEEE Transactions on Neural Networks and Learning Systems (TNNLS),  % 2017
	IEEE Global Conference on Signal and Information Processing (GlobalSIP),  % 2017
	IEEE Journal of Selected Topics in Signal Processing (J-STSP)  % 2016
}
% Professional Services: reviewer, editor, chair, organizer?, thesis committees?
\cvitem{Extra}{Brass Band musician, militia firefighter officer, computing systems and open-source enthusiast.}

%\section{Extra-Curricular}  % Interests, Extra-Curricular Activities
% TODO: Keep or remove? In Miscellaneous?
%\cvlistitem{Brass Band musician, playing in two bands at the Swiss national championship}
%\cvlistitem{Firefighter officer in the local militia}
%\cvlistitem{Computing systems and open-source enthusiast}
%\cvlistitem{Interests in entrepreneurship and start-ups}

%\section{Online}
% Only useful for short form without full listings (those links are otherwise prominently mentioned already).
% TODO: but shall I maintain a short form?
%\cvitemwithcomment{}{\httpslink{deff.ch}}{Publications, talks, teaching, advizing, software, projects.}
%\cvitemwithcomment{}{\httpslink{github.com/mdeff}}{Open source contributions.}
%\cvitemwithcomment{}{\httpslink{scholar.google.com/citations?user=Ztj2-gUAAAAJ}}{Publications.}
%\cvitemwithcomment{}{\httpslink{twitter.com/m_deff}}{Thoughts.}
% orcid, arxiv, zenodo, youtube, linkedin

\end{document}

% ---------------------------------------------------------

% \section{Features}
% from /usr/share/texmf-dist/tex/latex/moderncv/moderncv.cls

% \section{title}
% \subsection{title}

% \cventry[\spacing]{2000 -- 2010}{degree/title}{institution}{where}{grade}{%
% comment
% }

% \cvitem[1em]{header}{text}
% \cvitemwithcomment[1em]{header}{text}{comment}
% \cvdoubleitem[1em]{header1}{text1}{header2}{text2}
% \cvlistitem{item}
% \cvlistdoubleitem{item1}{item2}

% \begin{cvcolumns}
% 	\cvcolumn[0.4]{head}{content}
% 	\cvcolumn[0.4]{head}{content}
% \end{cvcolumns}
